\documentclass[11pt,a4paper,sans]{article}

\usepackage{graphicx}  
\usepackage{libertine}%  serif and sans serif
\usepackage[scaled=0.85]{beramono}%% mono
\usepackage{titlesec}
\usepackage[utf8]{inputenc}
\usepackage[normalem]{ulem}
\usepackage[usenames,dvipsnames]{color}
\usepackage{bbm}
\usepackage{bm}
\usepackage{flafter}	% keine Gleitobjekte vor der Position, an der sie definiert werden
\usepackage{color}
\usepackage[font=small,labelfont=bf,format=hang]{caption}[2017/02/28]		% Bildlegenden etc. anpassen
\usepackage{subfigure}		% für subfigures
\usepackage{pdfsync}	% erste llt Sync Datei für Sumatra Pdf Reader
\usepackage{array}
\usepackage{tabularx} 
\usepackage{longtable}	% kann Seitenumbrüche auch bei Tabellen
\usepackage{url}
\usepackage[a4paper,width=150mm,top=10mm,bottom=10mm]{geometry}

\setlength{\textwidth}{15cm}

\setlength{\headheight}{10pt}
\setcounter{secnumdepth}{3}
\setcounter{tocdepth}{3}

\sloppy 

%--------------------BEGIN DOCUMENT----------------------
\begin{document}

%--------------------TITLE-------------
\noindent\begin{minipage}{0.75\textwidth}% adapt widths of minipages to your needs
Carmeq Gmbh
\end{minipage}%
\hfill%
\begin{minipage}{0.2\textwidth}\raggedleft

\
end{minipage}

I am writing to apply for the postgraduate position beginning Fall 2012, as advertised on your department website. I am currently master student at the Moscow Institute of Physics and Technology in Dolgoprundy, Moscow Region, and fully expect to complete my master degree requirements by June 2012. I am highly interested in obtaining a PhD position at University of Amsterdam, as its physical research programs have excellent reputation that is known worldwide.

%I believe that my academic training and my 3 years of experience working in the the Institute of Radioengineering and Electronics of Russian Academy of Sciences prepare me to be an effective researcher in your department. My master diploma was conducted under the direction of Prof. Sergey Nikitov, and looks at the investigation of a relatively new phenomenon of the noncharge particles (magnons) anomalous Hall effect in heterogeneous ferromagnetic media. In my research, I developed theory of magnon transport in ferromagnetic materials in which exchange interaction can be neglected (in YIG for example) taking into account the demagnetizing field of the structure. The influence of the Berry phase effect on the dispersion of forward volume magnetostatic spin waves in periodic magnetic structures (magnonic crystals) was  discovered.  In such structures anomalous Hall effect increased tremendously near the Gamma point which can be useful for the applications in spintronics devices.

%I have extensive experience in experimental physics and achieved bachelor degree by experimental work in the field of parametric excitation of acoustic waves in nanostructured films of TbCo2/FeCo near the spin reorientation transition. Under the collaboration with Saratov State University I worked on the nanoimprint lithography setup (Obducat “Eitre”). I was acquainted with the full technological process of preparing the structure from the resist applying setup to cleaning samples setup in clean room. Several two-dimensional periodic structures samples on kvartz and nickel substrates were made with the resolution of 100 nm for further ferromagnetic resonance study. As well I had working experience on scanning electron microscope and scanning probe microscope (NT-MDT “Nanoeducator” and “Ntegra Spectra”) as a part of our laboratory preparation at MIPT.
%During my graduate training, I have been fortunate enough to also serve as a conference assistant on the 52th MIPT scientific conference and occasionally instruct students for some educational information on technical courses at my department .Through this participation, I have developed confidence and an interest in teaching and conference holding.

%Although my diploma thesis focuses on a single topic, other areas interest me for my future research from developing improved analytical models and methods for quasi particle dynamics at nanostructures in nonequilibrium states to experimental material studies by X-ray scattering, gas sorption, vibrational spectroscopy and electron and probe microscopy.

I look forward to the opportunity to link fundamental to applied research, participate in the organization of conferences and establish bonds between theory and experiment.
%I am sure that my theoretical background in solid state physics, well background in quantum mechanics and experimental preparation in physics of nanostructures will help me to solve various interdisciplinary problems at the Department of Physics of Ludwig-Maximilians-University Munich. 
I would enjoy discussing this position with you in the weeks to come. In the meantime, I am enclosing my Curriculum Vitae to this letter. If you require any additional materials or information, I am happy to supply it. Thank you very much for your consideration.

Cincerely yours, \\
\begin{minipage}{0.3\textwidth}
\textbf{Sergey Platonov} \\
\textit{Schlo{\ss}str.33 14059,\\
Berlin, Germany\\
+(49)-176-833-26-53\\
platonov.serge@gmail}
\end{minipage}

\end{document}